\chapter{The Linux host environment}
Before an engine can start up several configurations need to take place. Some of those configurations are made on the Linux host side and some
within the engine docker container. In any case the bootloader is the instance which is responsible to setup the environment and put it in a well
defined state such that the engine can funcation properly.
	\section{Filesystem layout of the Linux host}
	The filesystem layout is oriented on the filesystem hierarchy standard (FHS). In \emph{/srv} we find the following hierarchy:
	
	\tikzstyle{every node}=[draw=black,thick,anchor=west]
	\tikzstyle{selected}=[draw=red,fill=red!30]
	\tikzstyle{optional}=[dashed,fill=gray!50]
	\begin{tikzpicture}[
		grow via three points={one child at (0.5,-0.7) and two children at (0.5,-0.7) and (0.5,-1.4)},
		edge from parent path={(\tikzparentnode.south) |- (\tikzchildnode.west)}]
		\node(root) {/}
			child { node {srv}
				child { node {docker}}
				child { node {nas}}
				child { node {migration}}
				child { node {recommind}
					child { node {bootup}}
					child { node {pgbouncer}}
					child { node {projects}}
					child { node {axcng-web/service}}
				}
				child [missing] {}
				child [missing] {}
				child [missing] {}
				child [missing] {}
				child { node {tmp}}
			}
			child [missing] {}
			child [missing] {}
			child [missing] {}
			child [missing] {}
			child [missing] {}
			child [missing] {}
			child [missing] {}
			child [missing] {}
			child [missing] {}
			child { node {opt}
				child { node {recommind}
					child { node {bootup}}
				}
			}
			child [missing] {}
			child [missing] {}
			child { node {etc}}
			child { node {var}};
	\end{tikzpicture}
	
	We omit the presentation of directories which are not touched by us. As already said they follow the FHS. The other directories are maintained by us and contain the following contents.
	\paragraph{/srv:} \emph{/srv} contains data which are persistent by services. In our environment this means:
	\begin{description}
		\item[docker] Here is where the docker service stores the docker images and containers. It is usually not of important to us although it might run out of disk if too many docker images and containers are pulled and spawned respectively. However on normal operation the bootloader shutdown scripts take care to remove any container spawned earlier. The bootloader startup scripts in turn take care to spawn new containers as specified in their respective configuration files.
		\item[nas] In every POD a Windows-share server is being instantiated and which is used - for instance - to store production or csv exports. In order to give the engine process access to those server locations the POD specific Windows-share is mounted into this location and mapped into the engine container later on.
		\item[migration] If an engine server is migrated from the Windows to the Linux environment orchestration takes care about attaching the "old" Windows drives to the Linux instance. If that is the case the bootloader detects the Windows drives and mounts them read-only into this location. The bootloader within the engine container - once started - checks if there is need to migrate the contents on this location to the new one.
		\item[recommind] This is the location where everything is stored by our processes and which is being mapped into the engine container such that the engines do have access to it. The \emph{bootup} subdirectory contains the logfile produced by the bootloader which runs inside the engine container. If the engine container does not come up for what every reason this would be \textbf{the first place} to inspect to get an idea of what possibly went wrong during engine docker bootup. The \emph{pgbouncer} subdirectory is being used by the pgbouncer container. The bootloader on the host takes care such that the pgbouncer configuration is written to this directory. The \emph{projects} directory is our \emph{\$MINDSERVER\_PROJECTS} directory and contains every project plus the launcher service and processcontrol content. On the app-servers the \emph{axcng-web} and \emph{axcng-service} directory are located here too but should usually not of the concern of the index team.
		\item[tmp] First: this location has \textbf{nothing} to do with the temp drive/directory of the Linux host. For engine servers dedicated temp-drives are created and attachted to the engine instance. Those drives are mount to this location and mapped into the docker container later on to provide a dedicated temp drive large enough to hold every temporary data the engine container comes up with. It is used by the container which is provided by the docker service thus it is located in \emph{/srv}.
	\end{description}
	
	\paragraph{/opt:} contains the software packages which runs on the server. The only relervant one for the index team is located in the subdirectory \emph{recommind/bootup} which contains the bootloader which runs on the Linux host instance. Besides the bootloader itself it also contains the bootloader logfile(s) and is therefore the first place to take a look at when the Linux bootup procedure failed.
	
	\paragraph{/etc:} is the directory containing host specific and system-wide configuration files. Relervant for the index team are here the subdirectories
	\begin{description}
		\item[samba] which contains the samba configuration \emph{smb.conf} and which might be of intereset in case an operator has problems accessing the instances shared volume (which is \emph{/srv/recommind}).
		\item[systemd] which contains the \emph{systemd} unit-description files which specifies the \emph{systemd} endpoints to startup and shutdown recommind services (here shutdown phases of the Linux host bootloader).
	\end{description}
	
	\paragraph{/var:} is the directory containing every variable - i.e. continually changed - files during system uptime. Most relevant for problem analysis is the subdirectory \emph{log} which contains several logfiles especially \emph{syslog} among the others which is the one of the logfiles to inspect if something went wrong during system startup but also during system uptime.
	
	\section{Bootloader system configuration}
	The Linux host instances is being configured by the bootloader in various ways. Here we want to describe every phase in detail but omit the presentation of the phase specific source code. An detailed presentation of the Linux host bootloader source will be treated in chapter \ref{p01:ch08} on page \pageref{p01:ch08}.
	
		\subsection{\emph{bootup} phases}
			The \emph{bootup} phase will run at every instance startup. To ensure that one of the phases creates \emph{systemd} unit-descriptions and places them to \emph{/etc/systemd}. The \emph{systemd} init process in turn ensures that every such uni-description is read and produces a \emph{systemd} service endpoint. It will then show up by the \emph{systemctl} Linux command and can be started, stopped, restarted and queried for its status with that command. When enabled every thus endpoint is automatically started (in a specified order) by the \emph{systemd} init process at system startup.
			\subsubsection{prestart00}
			This phase is responsible for the following configurations:
			\begin{description}
				\item[Volumes] If a Linux instance is being created by the orchestration several AWS volumes might be created and attached to the instance. For every such volume a description in the bootloader configuration file exists which tells the volume setup code where the volume node is, in which logical volume group it depends and what filesystem it needs to carry.
				\begin{listing}[H]
					\caption{A sample volume description}
					\label{lst:p01:ch01:linhost_vol_desc}
					\inputminted{yaml}{\relative{chapter_01/section_3.2/volume_description_example.yaml}}
				\end{listing}
				This description is a 1:1 mapping as it would appear in the \emph{/etc/fstab} file.
			
				The \emph{block\_device} member tells us whether we're dealing with a block device. It is usually always true in the 			production environment but not necessarily in development or staging. For testing purposes it is possible to not specify this member or set it to \emph{False} and use some regular directory on disk and perform a bind-mount into another directory. For such volumes no volume setup is performed as it is already mounted.
			
				On the Linux host every volume which can be configured via the host yaml configuration file - except for the swap filesystems - will be put under control of the \emph{logical volume manager (lvm)}. The lvm allows us to define one volume consisting of several devices and which is of size equal to the sum of all those devices size. To structure thus logical volumes the set of devices form a \emph{group} for which we have to specify a \emph{group name} as an identifier. Once that is done we can assign this group to a \emph{logical volume}.
			 
				If multiple devices are configured for the same mount point its \emph{dirname} of the mount point specified by \emph{directory} defines the \emph{group name} the described device belongs to and the \emph{label} defines the \emph{logical volume} where the group will be assigned to.
			
				During volume setup the following rough steps are performed:
				\begin{itemize}
					\item seperate block devices from non-block devices
					\item import logical volumes which are already present
					\item create all volumes which are not already configured, put all devices into their respective volume groups and extend the the size of all logical volume by the sum of the newly free available size defined by every new device which was put into a volume group thats affecting those logical volumes. This is only performed on the set of block devices.
				\end{itemize}
			
				\item[Active Directory join] Every Linux engine is put into an active directory via \emph{sssd}. During AD join the relevant configuration options are read which specifies the \emph{realm}, \emph{user} - which is allowed to perform the join -, \emph{password} as well as the AD \emph{permitted groups} which are permitted to login to the Linux instance using their AD credentials.
				\begin{listing}[H]
					\caption{A sample ad realm description}
					\label{lst:p01:ch01:linhost_ad_realm_descr}
					\inputminted{yaml}{\relative{chapter_01/section_3.2/ad_realm_description_example.yaml}}
				\end{listing}
			
				\item[Logfile retention] Only relevant for app- and service-tier.
				\item[Lifecycle hook registration] Only relevant for app- and service-tier.
			\end{description}

			\subsubsection{prestart01}
			This phase is responsible for the following configuration:
			\begin{description}
				\item[Migration volume setup] During the migration of the Windows engine environment to the Linux engine environment we need to copy over the projects from a ntfs drive to the new Linux (xfs) drive. A usual setup in the Windows environment consists of  multiple disk which are configured into a logcial volume, named \emph{spanned volume}, and which provides a single disk of size of the sum of all disk which were specified to form the spanned volume.
			
			In the Linux environment the tool \emph{ldmtool} allows us to scan for such volumes and to create a device node for such a spanned volume in \emph{/dev/mapper} and which looks like \emph{ldm\_vol\_WIN-JF1F8KQJAPA-Dg0\_Volume}.
			
			During migration volume setup the bootloader tries to identifies a single ntfs or a spanned volume and mounts it into \emph{/srv/migration}. It fails in case there is more than one such volume detected as in this case the bootloader does not know which one to setup for migration.
			
			There is no Linux host configuration option needed for this. If the migration volume is not present the starting engine docker container knows that there is nothing to migrate.
			
				\item[Initialization of environment variables] Here every defined environment variable needed is put into \emph{/etc/environment}. The code always needs to know which environment variables shall make it into the system. There is no generalization mechanism. This also gives no place to inject unwanted environment variables. At the present time the configuration looks like this:
				\begin{listing}[H]
					\caption{A sample env var describtion}
					\label{lst:p01:ch01:linhost_env_var_descr}
					\inputminted{yaml}{\relative{chapter_01/section_3.2/env_var_description_example.yaml}}
				\end{listing}
			\end{description}
			
			\subsubsection{prestart10}
			In this phase the Windows file share (in recommind terminology: nas) and every volume configured in phase \emph{prestart00} is mounted into the system.
			\begin{description}
				\item[Mount of nas] The Windows nas is treated in a special way. It is not handled by the mount part of the bootloader since it might be the case that we need to protect the mount point from write access against other processes; here from the engine process to write to the nas mount point in case the nas could not be mounted into the system for what ever reason. Listing \ref{lst:p01:ch01:linhost_win_nas_descr} shows the nas configuration options.
				\begin{listing}[H]
					\caption{A sample env var describtion}
					\label{lst:p01:ch01:linhost_win_nas_descr}
					\inputminted{yaml}{\relative{chapter_01/section_3.2/win_nas_description_example.yaml}}
				\end{listing}
				The configuration specifies the AD user and password which is allowed to mount the nas into the system as well as the UNC path to the nas endpoint to be mounted.
				
				If - for whatever reason - the nas cannot be mounted into the system the mount point gets protected against write access via \emph{chattr} and \emph{chmod} to ensure that no process of the engine docker container is able to write to this location. This is to ensure that we do not loos whatever data should be written to that location. If we would allow to write into the engine docker containers nas location without the nas properly mounted into the system those data would only reside on the Linux host but would never be copied into the nas location. Assume that would be the case those data will be lost in case of an upgrade or patch process because those processes are designed to throw away the hole instance and create a new one as desired.
				
				\item[Mount of (other) volumes] Here every volume configured earlier in \emph{prestart00} is added to \emph{/etc/fstab} - if not already present - and mounted into the system to their respective mount points. 
				
				There is a special situation concerning the migration volume. The ntfs device driver implements the \emph{fuse (filesystem userspace)} api and thus resides in the userspace and not in the kernelspace. However as such it cannot be kept alive after the bootloader process lifetime ends and the init process is established. As there is no such parent process during system initialization the ntfs-3g device driver process cannot be inherited by the parent process of the terminated bootloader process. Hence the ntfs-3g terminates too and unmounts the Windows volume again. To be able to mount the Windows volume automatically the \emph{x-systemd.automount,nofail} option is set for this device forcing the systemd mount capability to mount our Windows volume automatically later on. This however forces us to implement in the engine container bootup some wait time to give the systemd service on Linux host side some time to trigger the mount.
			\end{description}
			
			\subsubsection{prestart50}
			Besides several configuration steps concerning filesystem setup and the embedding of external resources into the system (like the Windows share) we do also need to configure a set of additional agent and server processes on the Linux host to fullfil the requirement of exposing monitoring and directory shared information to other endpoints in the AWS environment. Those additional services and their configuration will now be explained.
			\begin{description}
				\item[pgbouncer configuration]
				\item[smb server configuration]
				\item[systemd unit description installation]
				\item[splunk config removal]
				\item[docker configuration update]
			\end{description}
			
			\subsubsection{prestart51}
			\subsubsection{start50}
			\subsubsection{poststart50}
		\subsection{\emph{shutdown} phases}
			\subsubsection{stop50}
			\subsubsection{poststop49}
			\subsubsection{poststop50}