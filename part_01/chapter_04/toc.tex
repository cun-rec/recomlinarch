\chapter{The docker architecture}\label{p01:ch5}
	\section{A short introduction to docker}
		\subsection{What is docker}
		Docker is a technology to containerize software applications together with their dependencies into a complete environment and thus isolating them from other processes. As such the docker hosts environment does not need to be poluted by software installations and configurations. It makes it very easy to deploy a software as everything needed to run the software is contained within the container. From a technological point of view docker uses the Linux kernel techniques like cgroups to limit the set of resources the encapsulated environment is able to access.
		\subsection{Images and Containers}
		Before we can run containers containing our software we first need to build a docker image. From that image a container can be instantiated which then runs everything we described in the environment we configured during image creation.
		\subsection{Docker description files}
		To create an image we use a \emph{\dockerfile{}} which describes how our environment looks like. For example if we would like to run an ssh daemon within a docker container (although it might not has much sense as you only can login \emph{into} the container not the host) a minimal \dockerfile{} would look like this:
		\begin{listing}[H]
			\caption{A sample \dockerfile{}}
			\label{lst:p01:ch05:sample_dockerfile}
			\inputminted{text}{\relative{chapter_04/section_5.1.2/Dockerfile}}
		\end{listing}
		\begin{itemize}
			\item[FROM] The \mintinline{text}{FROM} instruction tells us on what image the new one is based on (we don't create images from scratch thus we don't need to deal with creating a ``root'' image). Here it is based on an image with a minimal ubuntu 16.04 environment. 
			\item[RUN] The next \mintinline{text}{RUN} statement executes the commands which follow the instruction during the image build process. Here we update the package list, install the latest fixes, install the openssh server and finally create the pid-file location for the ssh daemon. In the second \mintinline{text}{RUN} statement we change the password for \rootuser{} to be \rootuser{} (this is a lightweight example not accounting for security. Never ever do this; this description is based on a minimal example for creating a horneypot and for that purpose it was desired to make the break into the system for an intruder as easy as possible).
			\item[EXPOSE] This statement will ensure that the port following to the instruction is exposed to the docker host. Remember that docker is a container-technology which isolates the environment we encapsulate. By default this also means that the ports used within the container are not exposed to the outside world.
			\item[CMD] This statement defines the command which shall be executed once the we run the image. In our case here we want the ssh daemon to start.
		\end{itemize}

		\subsection{Repositories}
		A repository is a location or endpoint an image is put into. This can be local as well as a remote endpoint. For better organization images we can be tagged/labeled. If not specified the default tag ist \mintinline{text}{latest}.
		\subsection{The most relevant docker commands}
		The docker container system is very complex. However most of the time the set of commands you will need to deal with is small. The in our opinion most relevant commands will be described here.
			\subsubsection{\mintinline{text}{docker images}}
			\mintinline{text}{docker images} gives us just a list of every image in our system and looks like this:
			\begin{listing}[H]
				\caption{Output of docker images}
				\label{lst:p01:ch05:docker_images}
				\inputminted{text}{\relative{chapter_04/section_5.1.2/docker_images.lst}}
			\end{listing}
			Here we see a list of three images. Note the first two images do have two different tags namely latest and a convention we use in our docker base repositories (see section \ref{p01:ch05:docker_base_repos} for an explaination about the naming convention). But it is the same image what you can see if you take a look at the image id. 
			\subsection{\mintinline{text}{docker pull}}
			We use the combination of the repository and the tag to uniquely identify a specific docker image. For example the first image id \mintinline{text}{745551d6bb3} could be pulled by executing \mintinline{text}{docker pull recommind/deployment-base:latest}. Of course we would need access to the repository \mintinline{text}{recommind/deployment-base}. Here in our example we don't have a address schema in our repository name as it is a local repository. To get a more complete example of how to login and pull images from a remote target please see section \ref{p01:ch05:container_service_login}
			\subsubsection{\mintinline{text}{docker build} and \mintinline{text}{docker tag}}
			To create a docker image we use the \mintinline{text}{docker build} command. The minimum arguments required is a so called \emph{build context} on the assumption that the \dockerfile{} resides in current working directory. Otherwise the \dockerfile{} can be refered to by the \mintinline{text}{-f} option. The \emph{build context} defines the scope with in the filesystem which is available for access to the \mintinline{text}{docker build} process. Since we're able to place files from the host system into the docker image during image creation the \mintinline{text}{docker build} process needs access to them. And since the \mintinline{text}{docker build} process is already a container process we need to specify a location which can be mapped into this process to give it access to files which are needed during image creation. The build process does \emph{not} have access to the whole system.
			
			Lets assume our \dockerfile{} is the one from listing \ref{lst:p01:ch05:sample_dockerfile}. The following listing shows the command we could use to build the docker image. We omitted parts of the \mintinline{text}{apt-get} output as it is a lot:
			\begin{listing}[H]
				\caption{Output of docker build}
				\label{lst:p01:ch05:docker_build}
				\inputminted{text}{\relative{chapter_04/section_5.1.2/docker_build.lst}}
			\end{listing}
			We told the build process to use the current working directory as our build context (by specifying \mintinline{text}{.} as last parameter to the command). So the first thing the build process is doing is to scan the build context to get aware of everything in it. As there is only the \dockerfile{} the process tells us the size of that file (as this is the size of data residing within the build context). Then the build process just executes every statement in the \dockerfile{}. Every such statement defines a step to be executed. Every such execution modifies the intermediate docker image. Like a version control system this modification will be committed hence we gain a new intermediate container. Like vcs's docker is aware of what changes were already committed if one step failed for whatever reason we can fix it, build again and the docker process will recognize that it only needs to apply the steps which were not committed already.
			
			So lets take a look at the first step. The first step tells the build process what image we're going to modify. Here we refered the image by \mintinline{text}{ubuntu:16.04}. Taking a look at the docker images in listing \ref{lst:p01:ch05:docker_images} shows that this image has id \mintinline{text}{f753707788c5}. So the build process takes this as our base. It than spawns a container of this image with id \mintinline{text}{451e73b4d1a5} and executes the statement. The result is a modified environment which the build process commits yielding a new intermediate docker image with id \mintinline{text}{6fab3ba17962}. As we don't need the spawned container anymore it tells us that he's going to remove that intermediate container. This procedure will be repeated until the \dockerfile{} has been processed completely. The last commit then defines the id of our image:
			\begin{listing}[H]
				\caption{List of docker images after docker build}
				\label{lst:p01:ch05:list_images_after_build}
				\inputminted{text}{\relative{chapter_04/section_5.1.2/list_images_after_build.lst}}
			\end{listing}
			We notice two things. First the new image listed has the image id which was reported in the success message of the build process. And second that the repository as well as the tag is specified as \mintinline{text}{none}. The reason is that we did not tell the build process to put the new image into a repository and tag it. We can catch up on this by tagging the image explicitely:
			\begin{listing}[H]
				\caption{Tagging an image}
				\label{lst:p01:ch05:tag_image}
				\inputminted{text}{\relative{chapter_04/section_5.1.2/tag_image.lst}}
			\end{listing}
			It is of course possible to tell the \mintinline{text}{docker build} process to tag the created image directly by specifying \mintinline{text}{-t reposirotyname:tag} as parameter.
			\subsubsection{\mintinline{text}{docker run} and \mintinline{text}{docker ps}}
			To spawn a container out of an image we use the \mintinline{text}{docker run} command. The easiest format is just
			\mintinline{text}{docker run <image_id>}. However, remember that we specified in the \dockerfile{} that the command to be executed is to start the sshd daemon and if not specified otherwise the run process will start the container in \emph{attached} mode i.e. we're attached to the container. But since the daemon is no shell the command just don't return until the sshd terminates. Fortunately we can start the container in \emph{dettached} mode by specifying the \mintinline{text}{-d} option. The run command has a whole bunch of options. We will not discuss every option here. You can get a list of all options by executing \mintinline{text}{docker help run}.
			
			Now to see that our container started successfully we just list the docker container processes. Here is how the run and ps list looks like:
			\begin{listing}[H]
				\caption{Running an image}
				\label{lst:p01:ch05:run_image}
				\inputminted{text}{\relative{chapter_04/section_5.1.2/run_image.lst}}
			\end{listing}
			When we run the image in dettach-mode the return value of the command is the id of the spawned container. Listing the running container processes confirms that we indeed launched a container from the image: just compare the used image id.
			\subsubsection{\mintinline{text}{docker stop} and \mintinline{text}{docker start}}
			We can start and stop containers using \mintinline{text}{docker start} and \mintinline{text}{docker stop} respectively. As each container gets a name as a synonym we can either specify the container by name or by its id. If we did not spawn the container giving it a name through the run command the name is choosen randomly.
			\begin{listing}[H]
				\caption{Start and stop containers}
				\label{lst:p01:ch05:start_stop_containers}
				\inputminted{text}{\relative{chapter_04/section_5.1.2/start_stop_containers.lst}}
			\end{listing}
			Both commands confirm its execution by printing the containers id or name respectively when they were successful.
			\subsubsection{\mintinline{text}{docker rm} and \mintinline{text}{docker rmi}}
			If we don't need a container and/or image anymore we can just remove them by invoking \mintinline{text}{docker rm} for removing containers and \mintinline{text}{docker rmi} to remove images:
			\begin{listing}[H]
				\caption{Removes containers/images}
				\label{lst:p01:ch05:remove_containers_images}
				\inputminted{text}{\relative{chapter_04/section_5.1.2/remove_containers_images.lst}}
			\end{listing}
			Like for starting and stopping the container you can specify the id or the name to \mintinline{text}{rm}. For \mintinline{text}{rmi} you can either specify the id or the combination of \mintinline{text}{repository:tag}.
			\subsubsection{\aws{} Container Service login}\label{p01:ch05:container_service_login}
			In some cases you need to work with the \aws{} container service to pull and/or push images from and to the docker registry to be able to test changes of the images. To do so you need several things:
			\begin{itemize}
				\item The secret and access key of your \aws{} account
				\item The permissions to interact with \aws{} container service registry
				\item The \aws{} client software package named \mintinline{text}{awscli}
			\end{itemize}
			The secret and access key should have been communicated to you on account creation. If you don't have them available contact the infrastructure team most likely John Bodenstein. To get permissions to interact with the \aws{} container service registry also contact John Bodenstein. The \aws{} client software can be installed by \mintinline{text}{sudo apt install awscli}. Once done execute \mintinline{text}{aws configure} and configure the \aws{} client (providing your secret and access key etc. pp.).
			
			To do a docker login to be able to use the \aws{} container service registry execute \mintinline{text}{aws ecr get-login --region us-east-1}. The command generates a \mintinline{text}{docker login} command line which you can use - i.e. execute - to perform the docker login into the \aws{} container service registry. Once done you can pull/push images by prefixing your image repository with the \aws{} container service registry address, namely \mintinline{text}{710991980648.dkr.ecr.us-east-1.amazonaws.com}. 
			
			For example take the repository list given by listing \ref{lst:p01:ch05:docker_images}. Lets assume we want to push our test image into the \aws{} container service registry. Lets furthermore assume we already performed the docker login as explained above. The first thing we need to do is to tag our test image into the same repository but on the \aws{} container service registry by executing
			\setmintedinline{breaklines=true}
			\begin{center}
				\mintinline{text}{docker tag recommind/test:test 710991980648.dkr.ecr.us-east-1.amazonaws.com/recommind/test:test}
			\end{center}
			\setmintedinline{breaklines=false}
			Now we specified where the location of the image shall be. We push the image into the repository by executing
			\begin{center}
				\mintinline{text}{docker push 710991980648.dkr.ecr.us-east-1.amazonaws.com/recommind/test:test}
                   \end{center}
                   Wait for the push to be complete.
                   
                   If you want to pull an image from the \aws{} container service registry just use the same repository specification as for the push command:
                   \begin{center}
                   		\mintinline{text}{docker pull 710991980648.dkr.ecr.us-east-1.amazonaws.com/recommind/test:test}
                 	\end{center}
	\section{The base docker mercurial repositories}\label{p01:ch05:docker_base_repos}
		\subsection{Naming conventions for docker repository names}
		\subsection{The \texttt{foundation} repository}
		\subsection{The \texttt{build\_base} repository}
		\subsection{The \texttt{deployment\_base} repository}
	\section{The docker images in cg}
		\subsection{The \texttt{build} repository}
		\subsection{The \texttt{deployment\_baseengine} repository}
		\subsection{The \texttt{deployment\_ingestionengine} repository}
		\subsection{The \texttt{deployment\_reviewengine} repository}
		\subsection{The \texttt{test} repository}
		\subsection{The cg docker gradle tasks}