\chapter{The Linux container environment}
	\section{Setting up CORE docker containers}
	In chapter \ref{p01:ch1} we mentioned the build pipeline. Now we describe the CORE installation procedure in more detail.

		\subsection{Filesystem layout of the engine Docker container}
		We find the following filesystem hierarchy within the engine docker container:
		
		\tikzstyle{every node}=[draw=black,thick,anchor=west]
		\tikzstyle{selected}=[draw=red,fill=red!30]
		\tikzstyle{optional}=[dashed,fill=gray!50]
		\begin{tikzpicture}[
			grow via three points={one child at (0.5,-0.7) and two children at (0.5,-0.7) and (0.5,-1.4)},
			edge from parent path={(\tikzparentnode.south) |- (\tikzchildnode.west)}]
			\node(root) {/}
				child { node {srv}
					child { node {migration}}
					child { node {recommind}
						child { node {bootup}}
						child { node {projects}}
					}
					child [missing] {}
					child [missing] {}
					child { node {tmp}}
				}
				child [missing] {}
				child [missing] {}
				child [missing] {}
				child [missing] {}
				child [missing] {}
				child { node {opt}
					child { node {recommind}
						child { node {bootup}}
					}
				}
				child [missing] {}
				child [missing] {}
				child { node {etc}}
				child { node {var}}
				child { node {WINDOWS-NAS-NAME}};
		\end{tikzpicture}
	
		The directories \emph{/srv/recommind}, \emph{/srv/migration} as well as the \emph{/WINDOWS-NAS-NAME} are shared between the Linux host and the docker container as mentioned in chapter \ref{p01:ch021} on page \pageref{p01:ch021} already.

		\subsection{Docker image types}
		Now we treat the different CORE docker image types and their responsibilities. The appropriate docker image files are located at \emph{integration/docker} in the CORE repository:
				
		\tikzstyle{every node}=[draw=black,thick,anchor=west]
		\tikzstyle{selected}=[draw=red,fill=red!30]
		\tikzstyle{optional}=[dashed,fill=gray!50]
		\begin{tikzpicture}[
			grow via three points={one child at (0.5,-0.7) and two children at (0.5,-0.7) and (0.5,-1.4)},
			edge from parent path={(\tikzparentnode.south) |- (\tikzchildnode.west)}]
			\node(root) {integration}
				child { node {docker}
					child { node {build}}
					child { node {deployment\_baseengine}
						child { node {ubuntu}
							child { node {install}}
						}
					}
					child [missing] {}
					child [missing] {}
					child { node {deployment\_ingestionengine}
						child { node {ubuntu}
							child { node {install}}
						}
					}
					child [missing] {}
					child [missing] {}
					child { node {deployment\_reviewengine}
						child { node {ubuntu}
							child { node {install}}
						}
					}
					child [missing] {}
					child [missing] {}
					child { node {test}
						child { node {ubuntu}
							child { node {execution}}
						}
					}
				};
		\end{tikzpicture}

		Those files are used to build the docker images. As there are dependencies between the different images we have to ensure that the necessary parent images exist before we build any image. As described in chapter \ref{p01:ch5} the \emph{build} docker image is based on \emph{build\_base} which doesn't live in the CORE repository. The same applies to \emph{deployment\_baseengine} which is a child of \emph{deployment\_base}. The CORE docker images have the following responsibility:
		\begin{description}\sloppy
			\item[build] The appropriate docker container is used for the build environment only. It provides the setup for the jenkins build job which performs the CORE java compile as part of the Ubuntu build.
			\item[deployment\_baseengine] This is a child image of \emph{deployment\_base}. Its parent is a base image for of all application specific docker containers like App server, CORE and Pgbouncer. Those containers usually contain java applications and have their own bootup logic which is implemented in python. Therefore the parent docker image file contains installation instructions for java, python and the required python modules. Furthermore \emph{deployment\_base} defines the bootup location \emph{/opt/recommind/bootup} where the bootup scripts can be found as well as a docker entrypoint in order to trigger those scripts after container startup.
			
			\emph{deployment\_baseengine} in turn is a parent of different CORE containers. The docker image file includes installation instructions in order to install the CORE zip. There are some installation scripts located at \emph{/integration/docker/deployment\_baseengine/ubuntu/install} which are copied into the docker container and triggered as part of the docker image build.
			\item[deployment\_ingestionengine] This is a child image of \emph{deployment\_baseengine}. Its install script copies the \emph{configuration.system.properties} file to the bootup source directory. This file contains the \emph{ingestion} role name which is consumed by the CORE bootup procedure.
			\item[deployment\_reviewengine] Same as mentioned in \emph{deployment\_ingestionengine} but we use the role \emph{reviewengine} and the install script installs the classic Axcelerate zip additionally.
			\item[test] This is a child image of \emph{deployment\_reviewengine}. It is used to run Ubuntu gate tests on the jenkins build machine. The \emph{execution} folder contains a script which takes the following arguments:
			\begin{itemize}
				\item --force\_run\_all: Optional flag to force the execution of all test suits even in case of failing suites.
				\item list of test suites: At least one gate test suit must be given. The paths are relative, e.g. \emph{gate/infrastructure/testsuite\_coreapi\_measures.xml}.
			\end{itemize}
			If any suite fails on the jenkins system, the \emph{mindserver-projects} folder will be archived and saved as \emph{mindserver-projects.zip} on \emph{laos} in an appropriate sub folder of \emph{.../CORE\_major\_minor/UbuntuGateTestLogs/build/distributions/test-output/...}. In addition we find a \emph{result.csv} file at this location which contains the failed suite names, the number of total tests per suite as well as the number of skipped and failed tests.
			
			Please note that the script consumes a \emph{TESTAGENT} zip file and the following environment variables:
			\begin{itemize}
				\item MINDSERVER\_HOME
				\item MINDSERVER\_PROJECTS
				\item JAVA\_HOME
				\item TEST\_VOLUME: The location where the \emph{test-output} folder will be created. Furthermore the script expects the \emph{TESTAGENT} zip file here.
			\end{itemize}
		\end{description}
		
		\subsection{The CORE bootup phases}\label{p01:ch0322}
		In chapter \ref{p01:ch221} we treated the different bootup phases on the Linux host. The CORE docker container contains its own bootup phases for the CORE part. The appropriate python code is located at \emph{bootup/bootup} in the CORE repository and consists of the following Linux specific phases. We omit the phases used by the \emph{batch\_asg} or any other stuff:
			
			\subsubsection{prestart00}
			\begin{description}\sloppy
				\item[check debug environment] This script alows debugging of CORE docker containers. If the container environment variable \emph{RECOM\_DEBUG} is set to an arbitrary value, the script blocks the container bootloader in the first prestart phase. This is useful as certain script failures may lead to a container bootup abortion. In this case we would be unable to log in to the stopped container in order to make manual code changes.
				\item[setup directories] This script creates the directories \emph{/srv/recommind/projects} and \emph{/srv/recommind/storage} with the desired permissions \emph{2775} and the user \emph{recomadm} of group \emph{recommind}. In addition, it sets ACL's on \emph{/srv/recommind}. Please note that \emph{/srv/recommind} has already been created on the docker host and ACL's have been set, too. We set the ACL's again during container bootup as we should support container testing. It provides the possibility of running CORE docker containers without the complete docker host environment.
			\end{description}
			\subsubsection{prestart50}
			\begin{description}
				\item[configure base] We have to create a CORE configuration map consisting of several configuration steps. This first script writes a \emph{remove} command to this map which is beeing filed with additional steps in subsequent prestart phases.
				\item[configure splunk] Writes a CORE specific splunk configuration input file by using an appropriate template. This file is beeing consumed by splunk which runs on the docker host.
				\item[setup base] This script prepares the mindserver projects and performs the corresponding basic setup. It is responsible for the following steps:
				\begin{itemize}
					\item Downloading the branding jar from S3 and renaming of branding icons
					\item Mindserver project preparation
					\item 64 Bit Crawler architecture setup
					\item Migrating Windows data if a migration volume is found
					\item Downloading the tomcat certificate from S3 if no migration volume is found and an empty project is in use
				\end{itemize}
			\end{description}
			\subsubsection{prestart60}
			\begin{description}
				\item[configure base] This is the second CORE configuration script which adds the two basic host roles \emph{Application Server} and \emph{Engine Server} as well as the process control identifier to the configuration map.
				\item[configure ingestion] Adds the \emph{Ingestion Engine} specific roles  \emph{Ingestion Application Server}, \emph{Ingestion Engine Server} and \emph{Crawler} to the CORE configuration map. This script is only executed for the \emph{ingestionengine} role.
				\item[configure review] Adds the \emph{Review Engine} specific roles  \emph{Axcelerate Application Server} and \emph{Axcelerate Engine Server} to the CORE configuration map. This script is only executed for the \emph{reviewengine} role.
				\item[setup launcherservice credentials] Optinal update of launcher service credentials by using the super user name and corresponding password which may be contained in the CORE yaml file.
				\item[setup mindserver process configuration] Sets the temp directory to \emph{/tmp} as well as optional JMX parameters defined in the CORE yaml file.
			\end{description}
			\subsubsection{prestart70}
			\begin{description}
				\item[coreconfiguration] As described in the previous prestart chapters we have built a CORE configuration map. This script consumes the map and performs the configuration.
			\end{description}
			\subsubsection{prestart80}
			\begin{description}
				\item[wait for master service] At the end of the prestart phase we have to wait for a running master service. The service is used by the CORE startup and certain poststart scripts which run against the service.
			\end{description}
			\subsubsection{start50}
			\begin{description}
				\item[start engine] After all \emph{prestart} phases have been executed successfully the launcher service process has to be started. As the sub process call may not block the following bootup phases this process has to run in the background.
			\end{description}
			\subsubsection{poststart60}
			\begin{description}
				\item[configure tomcat] This script has two responsibilities. First it has to download a keystore from Amazon S3 which is used for new projects. Secondly we have to configure tomcat at the master service. These steps are necessary for new projects and migrated data as well. The script is only executed on first bootup unless we had script failures. The appropriate tomcat parameters are configured by the following section of the CORE yaml file:
				
				\begin{listing}[H]
					\caption{Tomcat configuration}
					\label{lst:p01:ch02:core_tomcat_desc}
					\inputminted{yaml}{\relative{chapter_02/section_3.2/tomcat_example.yaml}}
				\end{listing}
				\item[fix windows paths] After a Linux migration we have to check all configuration fields for Windows specific paths which can be fixed automatically. The underlying CORE script needs master service access in order to iterate over those fields and to make configuration changes. The script fixes the following problems automatically:
				\begin{itemize}
					\item \emph{[A-Za-z]:\textbackslash Projects} will be changed to \emph{/srv/recommind/projects}
					\item \emph{[A-Za-z]:\textbackslash Storage} will be changed to \emph{/srv/recommind/storage}
					\item \emph{..\textbackslash Storage} will be changed to \emph{/srv/recommind/storage}
					\item \emph{\textbackslash \textbackslash Fileshare} will be changed to \emph{/Fileshare}. We mount the known Windows file share to the root of the docker container.
				\end{itemize}
					
					The fix tool writes a CSV file to \emph{/srv/recommind/migration/log}. This file provides information about the modified fields including the old and new value. In addition there is a column which shows whether the fields are Linux compatible. Please note that the fix tool can't fix everything automatically. It can't resolve arbitrary Windows paths for instance.
				\item[register pgbouncer] Registers pgbouncer at the master service. Pgbouncer runs in it's own docker container in parallel to the CORE container. It must be available before registration. CORE provides the appropriate registration script which isn't available within the pgbouncer container. The registration is only performed on first bootup and won't be repeated on further bootups unless we have registration failures.
			\end{description}
			\subsubsection{poststart70}
			\begin{description}
				\item[waitforlauncherservice] In the \emph{start} phase we've started the launcher service process. As this process runs in background we have to ensure that the docker container doesn't stop after the last bootup phase. The wait script blocks the last bootup phase until the launcher service process is finished. It gets the process information from the \emph{start} phase. Please make sure that this script lives in its own and last \emph{poststart phase}. If there are other scripts in the same phase, the bootloader won't be able to report errors of those scripts even thought all scripts within the same phase are executed in parallel. Please also note that any following bootup phases wouldn't be executed at all.
			\end{description}

	\section{Setting up PgBouncer docker containers}
	